\chapter{Zaključak}
S razvojem oblaka i aplikacija otvorenoga koda sve je veći broj firmi koji se bave razvojem
aplikacija, a pogotovo web baziranih aplikacija. Uvođenje sustava za kontinuirani razvoj aplikacija
omogućava bržu i konkurentniju izradu. Inženjeri koji rade s takvim sustavima dobivaju povratnu
informaciju češće i brže te se smanjuje kontekstno prebacivnaje koje smanjuje produktivnost. Samim
time firma kao cjelina postaje efektivnija i produktivnija. Moderne firme su dokaz da korištenje
takvih tehnologija ubrzava izradu novih funkcionalnosti. Firme i projekti koji nemaju sustav za
kontinuirani razvoj aplikacija su u opasnosti od konkurencije jer takvoj firmi treba puno više
vremena i resursa za izgradnju iste funkcionalnosti. Na primjer, firma Facebook je bila u
mogućnosti brže razvijati nove funkcionalnosti nego konkurencijska firma MySpace.

Korisnik ima na izboru velik broj komercijalnih i besplatnih, otvorenih rješenja. U ovom radu je
opisano i isprogramirano samo jedno od rješenja koje se temelji na otvorenom kodu. Takvo rješenje
može se koristiti za aplikacije otvorenog i zatvorenog (komercijalnog) koda u bilo kojem programskom
jeziku te na bilo kojoj Unix platformi. Opisani sustav će vjerojanto zadovoljit nove korisnike te
projekte s manjim brojem inženjera. S povećanjem kompleknosti projekta i broja inženjera potrebno je
doraditi sustav kako bi i dalje bio brz, konkurentan i adekvatan za firmu. Na primjer, uvođenje
sustava za plavo-zeleni razvoj (\textit{engl. blue-green deployment}) omogućuje pokretanje dvije
verzije aplikacija u istom trenutku te slanje određenog postotka zahtjeva na jednu od tih verzija.
Kakvo je ova tema izuzetno široka, nemoguće je projektirati idealno rješenje koje će zadovoljiti sve
korisnike.


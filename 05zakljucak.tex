\chapter{ZAKLJUČAK}
Razvojem oblaka i aplikacija otvorenoga koda sve je veći broj firmi koje se bave razvojem
aplikacija, a pogotovo web baziranih aplikacija. Uvođenje sustava za kontinuirani razvoj aplikacija
omogućava bržu i konkurentniju izradu. Inženjeri koji rade s takvim sustavima dobivaju povratnu
informaciju češće i brže te se smanjuje kontekstno prebacivanje koje smanjuje produktivnost. Samim
time firma kao cjelina postaje efektivnija i produktivnija. Moderne su firme dokaz da korištenje
takvih tehnologija ubrzava izradu novih funkcionalnosti. Firme i projekti koji nemaju sustav za
kontinuirani razvoj aplikacija u opasnosti su od konkurencije jer takvoj firmi treba puno više
vremena i resursa za izgradnju iste funkcionalnosti. Na primjer, firma Facebook bila je u mogućnosti
brže razvijati nove funkcionalnosti nego konkurencijska firma MySpace.

Korisnik ima na izbor velik broj komercijalnih i besplatnih, otvorenih rješenja. U ovome radu
opisano je i isprogramirano samo jedno od rješenja koje se temelji na otvorenome kodu. Takvo
rješenje može se koristiti za aplikacije otvorenog i zatvorenog (komercijalnog) koda u bilo kojem
programskom jeziku te na bilo kojoj Unix platformi. Opisani će sustav vjerojatno zadovoljiti nove
korisnike te projekte s manjim brojem inženjera. Povećanjem kompleksnosti projekta i broja
inženjera potrebno je doraditi sustav kako bi i dalje bio brz, konkurentan i adekvatan za firmu. Na
primjer, uvođenje sustava za plavo-zeleni razvoj (\textit{engl. blue-green deployment}) omogućava
pokretanje dvije verzije aplikacija u istome trenutku te slanje određenog postotka zahtjeva na jednu
od tih verzija.  Kako je ova tema izuzetno široka, nemoguće je projektirati idealno rješenje koje
će zadovoljiti sve
korisnike.


\chapter{Uvod}
Razvoj aplikacija uvelike se promjenio s razvojem novih tehnologija, a pogotovo u zadnjem desetljeću
s razvojem oblaka (engl.\textit{cloud}). Sve je veći broj poduzeća koji se bave razvojem aplikacija
te imaju potrebu što prije objaviti nadogradnje aplikacija. Stare metode objavljivanja aplikacija
traju dugo te vrijeme potrebno za objavu raste s rastom broja injženjera. Osim što su skupe,
podložne su i ljudskim greškama. Stoga se sve više poduzeća odlučuje za kontinuirani razvoj
aplikacija.

Korištenjem kontinuiranog razvoja aplikacija te brzog razvoja (engl.\textit{agile development})
moguće je objavljivati nadogradnje više puta dnevno~\citep{abrahamsson2017agile} bez potrebe za
inženjerom objave (engl.~\textit{release engineer}). Takav sustav mora biti u mogućnosti otkriti
promjene na sustavu za reviziju koda, pokrenuti testove jedinice, kompajlirati aplikaciju, provesti
dodatna testiranja te poslužiti takvu aplikaciju krajnjim korisnicima.

Poduzeće Facebook objavio je članak~\citep{rossi2016continuous} o korištenju kontinuiranog razvoja
gdje nepobitno dokazuju da brzina objavljivanja novog koda ne opada sa povećanjem broj injženjera
razvojnih timova i kvaliteta koda ne opada. Broj grešaka (engl.~\textit{bugs}) u kodu linearno raste
s brojem objavljivanja koda, kao što je i očekivano. Također, neovisno o broju inježenjera, u
prosjeku se pojavljuje jedna greška srednjeg prioriteta na oko 3000 pojedinačnih objava koda.
Facebook tvrdi da je jedan od razloga kontinuirano testiranje.

U drugom poglavlju opisana je jedna od infrastruktura kontinuiranog razvoja aplikacija. Opisan je
Git sustav za reviziju koda, Jenkins aplikacija za kontinuiranu integraciju programskog koda,
Docker za pohranjivanje slike aplikacije i Nginx za distribuciju i balansiranje zahtjeva korisnika.
U trećem poglavlju opisan je proces i izrada jednostavne internet aplikacije u Go programskom jeziku
zajedno s testovima jedinice i testovima funkcionalnosti. Opisana je i izrada Jenkins posla koja će
pokrenuti testove i kompajliranje te izraditi i spremiti Docker sliku. U četvrtom poglavlju opisan
je proces preuzimanje Docker te posluživanje korisnika preko Nginx programa.


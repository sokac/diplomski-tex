\chapter{UVOD}
Razvoj aplikacija uvelike se promijenio s razvojem novih tehnologija, a pogotovo u zadnjem
desetljeću s razvojem oblaka (\textit{engl. cloud}). Sve je veći broj tvrtki koje se bave razvojem
aplikacija te imaju potrebu što prije objaviti njihovu nadogradnje. Stare metode objavljivanja
aplikacija traju dugo te vrijeme potrebno za objavu raste s rastom broja inženjera zbog kompleknosti
koda i programa. Osim što su skupe, podložne su i ljudskim greškama. Stoga se sve više tvrtki
odlučuje za kontinuirani razvoj aplikacija.

Korištenjem kontinuiranog razvoja aplikacija te brzog razvoja (\textit{engl. agile development})
moguće je objavljivati nadogradnje više puta dnevno bez potrebe za inženjerom za objavu aplikacija
(\textit{engl. release engineer})~\citep{abrahamsson2017agile}. Takav sustav mora biti u mogućnosti
otkriti promjene na sustavu za reviziju koda, pokrenuti testove jedinice, kompajlirati aplikaciju,
provesti dodatna testiranja te poslužiti takvu aplikaciju krajnjim korisnicima. Sustav mora biti i u
mogućnost pravovremeno vratiti prethodnu verziju aplikacije.

Kontinuirani razvoj aplikacije zahtjeva određeni proces i arhitekturu. Potreban je sustav za
reviziju koda te standardiziran tijek rada, odnosno izmjene aplikacije. Bez standardiziranog tijeka,
nije moguće u potpunosti automatizirati proces izgradnje nove verzije. Zatim, potreban je i alat za
kontunuiranu integraciju koji pokreće određene zadatke nakon svake izmjene aplikacije. Isto tako,
potreban je i sustav koji može pokrenuti novu verziju aplikacije ili je napraviti dostupnom, ovisno
o tipu iste. U slučaju mrežne aplikacije, potrebno je prethodnu verziju aplikacije ukloniti nakon
određenog vremena.

Korporacija Facebook objavila je članak o korištenju kontinuiranog razvoja u kojem nepobitno
dokazuju da brzina i kvaliteta koda ne opada s povećanjem broja inženjera razvojnih
timova~\citep{rossi2016continuous}. Broj grešaka (\textit{engl. bugs}) u kodu linearno raste s
brojem objavljivanja koda, kao što je i očekivano. Također, neovisno o broju inženjera, u prosjeku
se pojavljuje jedna greška srednjeg prioriteta na oko 3000 pojedinačnih objava koda. Facebook tvrdi
da je jedan od razloga upravo kontinuirani razvoj aplikacija koji sadrži komponentu testiranja.

U drugom poglavlju opisana je korištena infrastruktura kontinuiranog razvoja aplikacija. Uspoređeni
su tipovi sustava za reviziju koda te njihov povijesni razvoj. Prikazane su osnove korištenja i
arhitektura izabranog sustava za reviziju koda -- Git. Zatim je opisan sustav za kontinuiranu
integraciju te osnove alata Jenkins. U istom poglavlju opisan je Docker -- alat za upravljanje
kontejnerima te je uspoređen s virtualizacijom, Nginx -- web server, Go -- programski jezik te
Cypress -- alat za integracijsko testiranje. U trećem poglavlju opisan je proces i izrada
jednostavne internet aplikacije u Go programskom jeziku zajedno s testovima jedinice i testovima
funkcionalnosti. Opisana je izrada Jenkins posla koji pokreće testove i kompajliranje te izradu i
spremanje Docker slike, program za menadžment Docker kontejnera koji je zadužen za pokretanje zadnje
verzije aplikacije, korištenje Nginx programa zaduženog za posluživanje HTTP (\textit{engl.
Hypertext Transfer Protocol}) zahtjeva, proces objavljivanja nove verzije aplikacije te proces
vraćanja na prethodnu verziju. Četvrto poglavlje zaključuje rad te opisuje moguće daljnje
nadrogradnje rada.

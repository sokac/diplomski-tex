\chapter{UVOD}
Razvoj aplikacija uvelike se promijenio s razvojem novih tehnologija, a pogotovo u zadnjem
desetljeću s razvojem oblaka (\textit{engl. cloud}). Sve je veći broj firmi koje se bave razvojem
aplikacija te imaju potrebu što prije objaviti njihovu nadogradnje. Stare metode objavljivanja
aplikacija traju dugo te vrijeme potrebno za objavu raste s rastom broja inženjera zbog kompleknosti
koda i programa. Osim što su skupe, podložne su i ljudskim greškama. Stoga se sve više firmi
odlučuje za kontinuirani razvoj aplikacija.

Korištenjem kontinuiranog razvoja aplikacija te brzog razvoja (\textit{engl. agile development})
moguće je objavljivati nadogradnje više puta dnevno~\citep{abrahamsson2017agile} bez potrebe za
inženjerom za objavu aplikacija (\textit{engl. release engineer}). Takav sustav mora biti u
mogućnosti otkriti promjene na sustavu za reviziju koda, pokrenuti testove jedinice, kompajlirati
aplikaciju, provesti dodatna testiranja te poslužiti takvu aplikaciju krajnjim korisnicima.

Firma Facebook objavila je članak o korištenju kontinuiranog razvoja u kojem nepobitno dokazuju da
brzina i kvaliteta koda ne opada s povećanjem broja inženjera razvojnih
timova~\citep{rossi2016continuous}. Broj grešaka (\textit{engl. bugs}) u kodu linearno raste s
brojem objavljivanja koda, kao što je i očekivano. Također, neovisno o broju inženjera, u prosjeku
se pojavljuje jedna greška srednjeg prioriteta na oko 3000 pojedinačnih objava koda. Facebook tvrdi
da je jedan od razloga kontinuirano testiranje.

% TODO - dodati tekst o samoj temi

U drugom poglavlju opisana je korištena infrastruktura kontinuiranog razvoja aplikacija. Uspoređeni
su tipovi sustava za reviziju koda te njihov povijesni razvoj. Prikazane su osnove korištenja i
arhitektura izabranog sustava za reviziju koda -- Git. Zatim je opisan sustav za kontinuiranu
integraciju te osnove alata Jenkins. U istom poglavlju opisan je Docker alat i uspoređen s
virtualizacijom; Nginx web server, Go programski jezik te Cypress alat za integracijsko testiranje.

U trećem poglavlju opisan je proces i izrada jednostavne internet aplikacije u Go programskom jeziku
zajedno s testovima jedinice i testovima funkcionalnosti. Opisana je izrada Jenkins posla koji
pokreće testove i kompajliranje te izradu i spremanje Docker slike; program za
menadžment Docker kontejnera koji je zadužen za pokretanje zadnje verzije aplikacije; korištenje
Nginx programa zaduženog za posluživanje HTTP zahtjeva; proces objavljivanja nove verzije te proces
vraćanja na prethodnu verziju.

Četvrto poglavlje zaključuje rad.
